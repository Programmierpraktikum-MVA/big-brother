\documentclass[xcolor=x11names,table]{beamer}

\usepackage{myDefaultPackageSetup/Beamer/myDefaultsBeamer}
\include{myDefaultPackageSetup/Beamer/myTheme1}

%%%%%%%%%%%%%%%%%%%%%%%%%%%%%%%%%%%%%
%       Show notes on pympress      %
%%%%%%%%%%%%%%%%%%%%%%%%%%%%%%%%%%%%%
%\usepackage{pgfpages}
%\setbeamertemplate{note page}[plain]
%\setbeameroption{show notes on second screen=right}
%%\setbeameroption{show only notes}

\title[\textcolor{white}{Third Milestone}]
{Group A3 - Big Brother: Third Milestone}
\author{Group A3}
\institute[TU Berlin]{TU Berlin}
\date{\today}

\begin{document}
\begin{frame}
\titlepage
\end{frame}

%\begin{frame}{<++>}
%\tableofcontents
%\end{frame}

\section{Video training (eduVid)}

\begin{frame}{Motivation \& Overview}
\begin{itemize}
    \item Step 1: Summary of the issues of recorded classes in the form of
    indexes.
    \item Step 2: Informed learning.
    \item Step 3: Conveniently switch between specific topics on 
    which the student wants to focus more or repeat.
\end{itemize}
\end{frame}

\begin{frame}{Python modules}
\begin{itemize}
    \item OpenCV
    \item Tesseract
    \item The Natural Language Toolkit (NLTK)
    \item Rapid Automatic Keyword Extraction (RAKE)
    \item VLC
\end{itemize}
\end{frame}

\begin{frame}{Pipeline process}
\begin{enumerate}
    \item Script extraction
    \item Summarization
    \item Extraction of keywords
    \item Recognizing the change of presentation slides and recognizing
    the content on them using OCR
    \item If keyword from script $=$ keyword from presentation -
    index
    \item Access to relevant segments by demand
    \item Use a knowledge graph to make the program work more efficiently
    - (keywords are not always used when talking about a given topic)
\end{enumerate}
\end{frame}

\begin{frame}{Demo}
We are going to demonstrate the following functionalities:
\begin{itemize}
    \item video footage processing (from presentation till text)
    \item video player
\end{itemize}
\end{frame}

\setbeamercolor{background canvas}{bg=blue!20}
\begin{frame}{Any Questions?}
\end{frame}

\end{document}

